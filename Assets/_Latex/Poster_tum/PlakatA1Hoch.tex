%%%%%%%%%%%%%%%%%%%%%%%%%%%%%%%%%%%%%%%%%%%%%%%%%%%%%%%%%%%%%%%%%%%%%%%%%%%%%%%%
% TUM-Vorlage: Plakat A1 Hochformat
%%%%%%%%%%%%%%%%%%%%%%%%%%%%%%%%%%%%%%%%%%%%%%%%%%%%%%%%%%%%%%%%%%%%%%%%%%%%%%%%
%
% Rechteinhaber:
%     Technische Universität München
%     https://www.tum.de
% 
% Gestaltung:
%     ediundsepp Gestaltungsgesellschaft, München
%     http://www.ediundsepp.de
% 
% Technische Umsetzung:
%     eWorks GmbH, Frankfurt am Main
%     http://www.eworks.de
%
%%%%%%%%%%%%%%%%%%%%%%%%%%%%%%%%%%%%%%%%%%%%%%%%%%%%%%%%%%%%%%%%%%%%%%%%%%%%%%%%

%%%%%%%%%%%%%%%%%%%%%%%%%%%%%%%%%%%%%%%%%%%%%%%%%%%%%%%%%%%%%%%%%%%%%%%%%%%%%%%%
\input{./Ressourcen/Plakat/PraeambelA1Hoch.tex} % !!! NICHT ENTFERNEN !!!
\input{./_Einstellungen.tex}                    % !!! DATEI ANPASSEN !!!
%%%%%%%%%%%%%%%%%%%%%%%%%%%%%%%%%%%%%%%%%%%%%%%%%%%%%%%%%%%%%%%%%%%%%%%%%%%%%%%%


\newcommand{\PlakatTitel}{Torwandschießen}


%%%%%%%%%%%%%%%%%%%%%%%%%%%%%%%%%%%%%%%%%%%%%%%%%%%%%%%%%%%%%%%%%%%%%%%%%%%%%%%%
\input{./Ressourcen/Plakat/Anfang.tex} % !!! NICHT ENTFERNEN !!!
%%%%%%%%%%%%%%%%%%%%%%%%%%%%%%%%%%%%%%%%%%%%%%%%%%%%%%%%%%%%%%%%%%%%%%%%%%%%%%%%


%%%%%%%%%%%%%%%%%%%%%%
%% 2-Spalten-Layout %%
%%%%%%%%%%%%%%%%%%%%%%

\PlakatKopfzeileLeer
\PlakatFusszeile{%
    \textbf{\UniversitaetName}\\
    \FakultaetName\\
    \LehrstuhlName%
}

\PlakatTitelEins{\PlakatTitel}
\PlakatTitelZwei{Marcel Bruckner, Kevin Bein, Jonas Schulz, Chandruscm}
\PlakatTitelDrei{Technical University Munich}

\begin{multicols*}{2}

%\setlength{\PlakatBeschreibungBeispielbildBeschnitt}{65cm} % Anpassen der Größe des Beispielbildes

\PlakatUeberschrift{Abstract}

The goal of this project is to recreate the popular german game \textit{Torwandschießen} in VR. 

\PlakatUeberschrift{Game}

In \textit{Torwandschießen}, the player tries to kick a football through two openings in a wall.
One hole is aligned at the bottom left and the other hole is at the top right.
The player has five tries to kick the Ball through either of the two openings and collect points (1 point for the bottom and 3 points for the top hole).
Both holes are just slightly bigger than the diamter of the ball which makes it difficult to hit.
When either of the two holes is hit by the ball, the corresponding points are accredited.
The game ends after five tries.

\PlakatUeberschrift{Overview}

A Microsoft Kinect Sensor is used to track the player and his movements.
From this input data and the kinect SDK, a body skeleton is calculated and extracted (TODO: figure 2).
By using \textit{Linear Blend Skinning}, this skeleton is mapped onto the 3D Model of a football player in Unity.
For this game, it is especially important to detect movements of the lower body: the player's hips, knees, legs and feet.
With the movement mapped to the model, the only thing left is to construct hitboxes around the players feet.

The scene consists of the Torwand and a ball, both which are hittable.
Since the collision detection and physics calculation is not part of this project, Unity is used here again. 
Still, the model parameters, the position of the ball relative to the player, the hitbox sizes and shapes as well as the velocity calculation for when the player hits the ball need to be optimized and tuned manually.
With this setup, the only thing left is for the player to hit the ball in a good angle and accelerate it towards the Torwand.

\PlakatUeberschrift{Linear Blend Skinning}

\columnbreak
\PlakatBild[0cm \PlakatBeschreibungBeispielbildBeschnitt{} 0cm 0cm]{./Ressourcen/_Bilder/sample_image.png}{Unity Scene Rendering}

\end{multicols*}

\clearpage


%%%%%%%%%%%%%%%%%%%%%%
%% 3-Spalten-Layout %%
%%%%%%%%%%%%%%%%%%%%%%

%\PlakatKopfzeileMitDreizeiler
%\PlakatFusszeileLeer
%
%\PlakatTitelEins{\PlakatTitel}
%\PlakatTitelZwei{Überschrift 2 läuft über gesamte Papierbreite}
%\PlakatTitelDrei{Überschrift 3 läuft über gesamte Papierbreite oder gemäß Spaltenbreite}
%
%\begin{multicols*}{3}
%
%\input{./PlakatBeschreibung.tex}
%
%
%\PlakatUeberschrift{Blindtext}
%
%\lipsum[1-4]
%
%
%\end{multicols*}
%
%\clearpage


%%%%%%%%%%%%%%%%%%%%%%
%% Bild-Layout %%
%%%%%%%%%%%%%%%%%%%%%%

%\PlakatKopfzeileMitEinzeiler
%\PlakatFusszeileMehrspaltig{
%    Hier kann ein längerer Text stehen, der in mehreren Spalten angeordnet wird.
%    \vfill\columnbreak
%    Durch \texttt{\textbackslash{}vfill\textbackslash{}columnbreak} lässt sich ein Spaltenwechsel erzwingen.
%    \vfill\columnbreak~
%    \vfill\columnbreak~
%}
%
%\PlakatTitelEins{\PlakatTitel}
%\PlakatTitelZwei{Überschrift 2 läuft über gesamte Papierbreite}
%
%\PlakatBildGanzseitig[0cm 8cm 0cm 0cm]{./Ressourcen/_Bilder/SternenhimmelHochkant.jpg}{Bildunterschrift, Autor etc}
%
%\clearpage


%%%%%%%%%%%%%%%%%%%%%%%%%%%
%% Layout für wenig Text %%
%%%%%%%%%%%%%%%%%%%%%%%%%%%

%\PlakatKopfzeileMitEinzeiler
%\PlakatFusszeileLeer
%
%\PlakatTitelSchriftgroesse
%
%Wählen Sie für Plakate mit wenig Text gerne eine größere Schrift (siehe Auswahl für Überschriften). Achten Sie bitte darauf, dass die Schriftgröße Zweck und Gebrauch des Plakats entspricht.


%%%%%%%%%%%%%%%%%%%%%%%%%%%%%%%%%%%%%%%%%%%%%%%%%%%%%%%%%%%%%%%%%%%%%%%%%%%%%%%%
\end{document} % !!! NICHT ENTFERNEN !!!
%%%%%%%%%%%%%%%%%%%%%%%%%%%%%%%%%%%%%%%%%%%%%%%%%%%%%%%%%%%%%%%%%%%%%%%%%%%%%%%%
