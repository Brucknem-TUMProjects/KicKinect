%%%%%%%%%%%%%%%%%%%%%%%%%%%%%%%%%%%%%%%%%%%%%%%%%%%%%%%%%%%%%%%%%%%%%%%%%%%%%%%%
% TUM-Vorlage: Plakat A0 Querformat
%%%%%%%%%%%%%%%%%%%%%%%%%%%%%%%%%%%%%%%%%%%%%%%%%%%%%%%%%%%%%%%%%%%%%%%%%%%%%%%%
%
% Rechteinhaber:
%     Technische Universität München
%     https://www.tum.de
% 
% Gestaltung:
%     ediundsepp Gestaltungsgesellschaft, München
%     http://www.ediundsepp.de
% 
% Technische Umsetzung:
%     eWorks GmbH, Frankfurt am Main
%     http://www.eworks.de
%
%%%%%%%%%%%%%%%%%%%%%%%%%%%%%%%%%%%%%%%%%%%%%%%%%%%%%%%%%%%%%%%%%%%%%%%%%%%%%%%%

%%%%%%%%%%%%%%%%%%%%%%%%%%%%%%%%%%%%%%%%%%%%%%%%%%%%%%%%%%%%%%%%%%%%%%%%%%%%%%%%
%\input{./Ressourcen/Plakat/PraeambelA0Quer.tex} % !!! NICHT ENTFERNEN !!!
\input{./Ressourcen/Plakat/PraeambelA1Quer.tex} % !!! NICHT ENTFERNEN !!!
\input{./_Einstellungen.tex}                    % !!! DATEI ANPASSEN !!!
%%%%%%%%%%%%%%%%%%%%%%%%%%%%%%%%%%%%%%%%%%%%%%%%%%%%%%%%%%%%%%%%%%%%%%%%%%%%%%%%


\newcommand{\PlakatTitel}{Kickinect - VR Torwandschießen}


%%%%%%%%%%%%%%%%%%%%%%%%%%%%%%%%%%%%%%%%%%%%%%%%%%%%%%%%%%%%%%%%%%%%%%%%%%%%%%%%
\input{./Ressourcen/Plakat/Anfang.tex} % !!! NICHT ENTFERNEN !!!
%%%%%%%%%%%%%%%%%%%%%%%%%%%%%%%%%%%%%%%%%%%%%%%%%%%%%%%%%%%%%%%%%%%%%%%%%%%%%%%%


%%%%%%%%%%%%%%%%%%%%%%
%% 3-Spalten-Layout %%
%%%%%%%%%%%%%%%%%%%%%%

\PlakatKopfzeileLeer
\PlakatFusszeile{%
    \textbf{\UniversitaetName}\\
    \FakultaetName\\
    \LehrstuhlName%
}

\PlakatTitelEins{\PlakatTitel}
%\PlakatTitelZwei{Überschrift 2 läuft über gesamte Papierbreite}
\PlakatTitelDrei{Marcel Bruckner, Kevin Bein, }

\begin{multicols*}{3}

% Examples
%\input{./PlakatBeschreibung.tex}

\PlakatUeberschrift{Abstract}

The goal of this project is to recreate the popular german game \textit{Torwandschießen} in VR. 

\PlakatUeberschrift{Game}

In \textit{Torwandschießen}, the player tries to kick a football through two openings in a wall.
One hole is aligned at the bottom left and the other hole is at the top right.
The player has five tries to kick the Ball through either of the two openings and collect points (1 point for the bottom and 3 points for the top hole).
Both holes are just slightly bigger than the diamter of the ball which makes it difficult to hit.


\PlakatUeberschrift{Implementation}

A Microsoft Kinect Sensor is used to track the player and his movements.
From this input data, a body skeleton is calculated and extracted (TODO: figure 2).
By using \textit{Linear Blend Skinning}, this skeleton is mapped onto the 3D Model of a football player.
For this game, it is especially important to detect movements of the lower body: the player's legs and foots.
Hitboxes are constructed around the players feet and the ball. 
If both collide in the virtual reality, we calculate and apply a velocity to the ball and accelarate it towards the Torwand.
When either of the two holes is hit perfectly, the corresponding points are accredited.
The game ends after five tries.

\end{multicols*}

\clearpage


%%%%%%%%%%%%%%%%%%%%%%
%% 5-Spalten-Layout %%
%%%%%%%%%%%%%%%%%%%%%%

%\PlakatKopfzeileMitDreizeiler
%\PlakatFusszeileLeer
%
%\PlakatTitelEins{\PlakatTitel}
%\PlakatTitelZwei{Überschrift 2 läuft über gesamte Papierbreite}
%\PlakatTitelDrei{Überschrift 3 läuft über gesamte Papierbreite oder gemäß Spaltenbreite}
%
%\begin{multicols*}{5}
%
%\setlength{\PlakatBeschreibungBeispielbildBeschnitt}{15cm} % Anpassen der Größe des Beispielbildes
%\input{./PlakatBeschreibung.tex}
%
%
%\PlakatUeberschrift{Blindtext}
%
%\lipsum[1-8]
%
%
%\end{multicols*}
%
%\clearpage


%%%%%%%%%%%%%%%%%%%%%%
%% 1-Bild-Layout    %%
%%%%%%%%%%%%%%%%%%%%%%

%\PlakatKopfzeileMitEinzeiler
%\PlakatFusszeileMehrspaltig{
%    Hier kann ein längerer Text stehen, der in mehreren Spalten angeordnet wird.
%    \vfill\columnbreak
%    Durch \texttt{\textbackslash{}vfill\textbackslash{}columnbreak} lässt sich ein Spaltenwechsel erzwingen.
%    \vfill\columnbreak~
%    \vfill\columnbreak~
%    \vfill\columnbreak~
%    \vfill\columnbreak~
%    \vfill\columnbreak~
%    \vfill\columnbreak~
%    \vfill\columnbreak~
%}
%
%\PlakatTitelEins{\PlakatTitel}
%\PlakatTitelZwei{Überschrift 2 läuft über gesamte Papierbreite}
%
%\PlakatBildGanzseitig[0cm 18cm 0cm 0cm]{./Ressourcen/_Bilder/SternenhimmelQuer.jpg}{Bildunterschrift, Autor etc}
%
%
%\clearpage


%%%%%%%%%%%%%%%%%%%%%%%%%%%
%% Layout für wenig Text %%
%%%%%%%%%%%%%%%%%%%%%%%%%%%

%\PlakatKopfzeileMitEinzeiler
%\PlakatFusszeileLeer
%
%\PlakatTitelSchriftgroesse
%
%Wählen Sie für Plakate mit wenig Text gerne eine größere Schrift (siehe Auswahl für Überschriften). Achten Sie bitte darauf, dass die Schriftgröße Zweck und Gebrauch des Plakats entspricht.
%

%%%%%%%%%%%%%%%%%%%%%%%%%%%%%%%%%%%%%%%%%%%%%%%%%%%%%%%%%%%%%%%%%%%%%%%%%%%%%%%%
\end{document} % !!! NICHT ENTFERNEN !!!
%%%%%%%%%%%%%%%%%%%%%%%%%%%%%%%%%%%%%%%%%%%%%%%%%%%%%%%%%%%%%%%%%%%%%%%%%%%%%%%%
